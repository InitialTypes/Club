\documentclass[xcolor={usenames,dvipsnames,svgnames}]{beamer}

%%% Packages %%%

% Use T1 and a modern font family for better support of accents, etc.
\usepackage[T1]{fontenc}
\usepackage[utf8]{inputenc}
%\usepackage{lmodern}  % Latin Modern (successor of Computer Modern)
\usepackage{palatino}  % Palatino

\usefonttheme[onlymath]{serif}

\mode<presentation>
{
  \usetheme{Singapore}
  \usecolortheme{rose}
  %\setbeamercovered{transparent}
  \setbeamercovered{invisible}
  \setbeamercolor*{alerted text}{parent=titlelike}
  % get rid of all of the navigation stuff
  \beamertemplatenavigationsymbolsempty{}

  % slide number in the lower-right, in gray
  \setbeamertemplate{footline}{%
    \raisebox{5pt}{\makebox[\paperwidth]{%
        \hfill\makebox[20pt]{\color{gray}%
        \scriptsize\insertframenumber}}}\hspace*{5pt}}

  % use standard frame number
  %\setbeamertemplate{footline}[frame number]
}

% Use colored text to emphasize things.
\renewcommand{\emph}[1]{\alert{#1}}

% For author-year citations.
%\usepackage{natbib}

% Language support
\usepackage[english]{babel}

% Support for easily changing the enumerator in
% enumerate-environments.
%\usepackage{enumerate}

% Support for importing images
%\usepackage{graphicx}

% Use hyperlinks
\usepackage{hyperref}

% Don't load xcolors package in beamer: use document class option
% instead...
%\usepackage[usenames,dvipsnames]{xcolor}

% AMS Math typesetting
\usepackage{amsmath}    % AMS math basics
\usepackage{amsfonts}   % Math fonts
\usepackage{amssymb}    % Math symbols
\usepackage{mathtools}  % Extensible arrows etc.

% More math symbols (e.g. double-brackets: \llbracket, \rrbracket)
%\usepackage{stmaryrd}

% A nice monospace font for listings, etc.
\usepackage[scaled]{beramono}
%\usepackage{inconsolata}

% Conditionals
\usepackage{xifthen}

% Proof trees
\usepackage{bcprules}
\usepackage{bussproofs}

% Make rule labels smaller (for rules typeset with bcprules)
\newif\ifsmallrulenames
\smallrulenamestrue

% Style used in trees typeset with bussproofs (to be consistent with
% style used in rule definitions).
\newcommand{\rulelabel}[1]{{\small\sc (#1)}}

% Multiple columns.
\usepackage{multicol}

%%% Macro definitions

% Helper macro that returns #2 if #1 is empty, #3 otherwise.  The
% macro will misbehave if #1=\\.
\newcommand{\ifempty}[3]{\ifx\\#1\\ #2\else #3\fi}

% Use blue-ish color when shading.
\definecolor{shade}{rgb}{0.83, 0.8, 1.0}

% Highlights parts of equations/typing rules only on some slides.
\newcommand<>\hlbox[1]{%
  \alt#2{%
    \colorbox{shade}{\ensuremath{#1}}}{%
    \colorbox{white}{\ensuremath{#1}}}}

% A variant of hlbox that does not have extra margins.
\newcommand<>\shlbox[1]{%
  \alt#2{%
    \phantom{#1}\makebox[\fboxsep]{}%
    \mathchoice{%
      \smash{\llap{\colorbox{shade}{$\displaystyle#1$}}}}{%
      \smash{\llap{\colorbox{shade}{$\textstyle#1$}}}}{%
      \smash{\llap{\colorbox{shade}{$\scriptstyle#1$}}}}{%
      \smash{\llap{\colorbox{shade}{$\scriptscriptstyle#1$}}}}%
    \makebox[-\fboxsep]{}}{#1}}

% Colors for terms, variables, substitutions, etc.
\colorlet{varc}{PineGreen}
\colorlet{termc}{Blue}
\colorlet{subc}{Maroon}

% Handy for coloring terms
\renewcommand{\v}[1]{\textcolor{varc}{#1}}
\renewcommand{\t}[1]{\textcolor{termc}{#1}}
\newcommand{\s}[1]{\textcolor{subc}{#1}}

% Separator in syntax definitions
\newcommand{\DefOr}{\ensuremath{\hspace{0.2em}\big|\hspace{0.2em}}}


%%% Macros for constructors (types, terms, substitutions, judgments)

% NOTE. It's a bit cumbersome to use these macros when typesetting
% syntax and rules because all one sees in the TeX code is a salad of
% curly braces.  However, there's a payoff at the end because we can
% redefine the macros to change the "interpretation" of the terms and
% judgments (as morphism typings, sequents, etc.) simply be redefining
% the macros.  See the last few slides for examples.

% Type and term constructors
\newcommand{\base}[1][]{\mathrm{b}}
\newcommand{\unit}{\mathbf{1}}
\newcommand{\fun}[2]{#1 \to #2}
\newcommand{\vz}{\v{0}}
\newcommand{\lam}[2]{\t{\lambda \ifempty{#1}{}{\v{#1}. \,} #2}}
\newcommand{\app}[2]{\t{{#1} \, {#2}}}
\newcommand{\appp}[3]{\app{\app{#1}{#2}}{#3}}
\newcommand{\apppp}[4]{\app{\appp{#1}{#2}{#3}}{#4}}

% Contexts
\newcommand{\cempty}{\cdot}
\newcommand{\cext}{,}

% Substitutions
\newcommand{\subst}[2]{\textcolor{black}{\t{#1}[\s{#2}]}}
\newcommand{\sing}[2]{\s{\v{#1}\coloneqq \t{#2}}}
\newcommand{\id}[1][]{\s{\mathrm{id}}}
\newcommand{\wk}[1][]{\s{\uparrow}}
\newcommand{\pair}[2]{\s{\langle #1 , \t{#2} \rangle}}
\newcommand{\sempty}{\s{\langle \rangle}}
\newcommand{\scomp}[2]{\s{#1 \circ #2}}

% Free variables
\newcommand{\fv}{\mathsf{fv}}
\newcommand{\dom}{\mathsf{dom}}

% Parts of judgments
\newcommand{\tin}{:}
\newcommand{\tas}{\colon}  % for binders
\newcommand{\ts}{\vdash}

% Judgment forms
\newcommand{\bd}[2]{\v{#1} \tas #2}
\newcommand{\tp}[3]{#1 \ts \t{#2} \tin #3}
\newcommand{\sub}[3]{#1 \ts \s{#2} \tin #3}
\newcommand{\teq}[4]{#1 \ts \t{#2} = \t{#3} \tin #4}
\newcommand{\seq}[4]{#1 \ts \s{#2} = \s{#3} \tin #4}


%%% Beamer-related macros %%%

% Commands for disabling numbering of backup slides.
% Adapted from https://stackoverflow.com/a/2370399
\newcounter{finalframe}
\newenvironment{backupslides}{%
  % Save the frame counter before starting the backup slides.
  \setcounter{finalframe}{\value{framenumber}}
}{%
  % Restore the frame counter to its value before the backup slides.
  \setcounter{framenumber}{\value{finalframe}}
}


%%% Document info %%%

\title{The Simply Typed Lambda Calculus}

\subtitle{and Variants Thereof}

\author{Sandro~Stucki}

\institute[GU]{%
  University of Gothenburg~|~Chalmers, Sweden}

\date{\footnotesize%
  Virtual ITC Lecture 1 ~~--~~ 5 Nov 2020\\[4mm]%
  \url{sandro.stucki@gu.se}\hspace{5mm}\url{@stuckintheory}\\[5mm]%
  \includegraphics[height=16mm]{img/gu-logo}\hspace{10mm}%
  \includegraphics[height=12mm]{img/chalmers-logo}
}


% To show the TOC at the beginning of each section, uncomment this:
% \AtBeginSection[]
% {
%   \begin{frame}<beamer>{Outline}
%     \tableofcontents[currentsection]
%   \end{frame}
% }

% To show the TOC at the beginning of each subsection, uncomment this:
% \AtBeginSubsection[]
% {
%   \begin{frame}<beamer>{Outline}
%     \tableofcontents[currentsection,currentsubsection]
%   \end{frame}
% }


% To uncover everything in a step-wise fashion, uncomment this:
%\beamerdefaultoverlayspecification{<+->}


%%% Start of the actual document %%%

\begin{document}

\begin{frame}%[noframenumbering]
  \titlepage
\end{frame}

\begin{frame}{On today's menu}
  %\tableofcontents
  \pause
  \begin{description}[<+->][Part III]
  \item[Part I] \hyperref[sec:stlc]{STLC}\\[.5em]

    The plain, old simply typed lambda calculus\\[.5em]

  \item[Part II] \hyperref[sec:explicit]{Explicit substitutions}\\[.5em]

    Adding explicit substitutions and their equational theory\\[.5em]

  \item[Part III] \hyperref[sec:debruijn]{De Bruijn indices}\\[.5em]

    Doing away with variables\dots\\[.5em]

  \item[Part IV] \hyperref[sec:CHL]{Correspondences}\\[.5em]

    Everything is connected.\\[.5em]
  \end{description}
\end{frame}

\section{STLC}
\label{sec:stlc}

\begin{frame}{Syntax}

  \begin{alignat*}{2}
    \mathcal{V} \quad \ni &\quad&
      \v{x}, \v{y}, \v{z}, &\dots
      \tag{Variable}\\
    \mathcal{T} \quad \ni &&
      \t{s}, \t{t}, \t{u} &::= \v{x} \DefOr \lam{x}{t} \DefOr \app{s}{t}
      \tag{Term}\\
    \mathcal{A} \quad \ni &&
      A, B, C &::= \base \DefOr A \to B
      \tag{Type}\\
    \mathcal{C} \quad \ni &&
      \Gamma, \Delta, E &::= \cempty \DefOr \Gamma \cext \bd{x}{A}
      \tag{Context}\\
  \end{alignat*}
\end{frame}


\begin{frame}{Term typing \hfill \fbox{$\tp{\Gamma}{t}{A}$}}
  \infax[Var]
  {\tp{\bd{x_n}{A_n}, \dotsc, \bd{x_0}{A_0}}{\v{x_i}}{A_i}}

  \infrule[Abs]
  {\tp{\Gamma \cext \bd{x}{A}}{t}{B}}
  {\tp{\Gamma}{\lam{x}{t}}{\fun{A}{B}}}

  \infrule[App]
  {\tp{\Gamma}{t}{\fun{A}{B}} \quad \tp{\Gamma}{u}{A}}
  {\tp{\Gamma}{\app{t}{u}}{B}}
\end{frame}


\begin{frame}{Substitutions}
  Substitutions are finite maps $\s{\sigma}\colon \mathcal{V} \to \mathcal{T}$
  from variables to terms.

  \bigskip

  Application of substitutions
  \begin{align*}
    \v{x}&[\s{\sigma}] \coloneqq \s{\sigma}(\v{x}) \\
    (\lam{x}{t})&[\s{\sigma}] \coloneqq
      \lam{x}{\subst{t}{\sigma\{\sing{x}{\v{x}}\}}} \\
    (\app{t}{u})&[\s{\sigma}] \coloneqq
      \app{\subst{t}{\sigma}}{\subst{u}{\sigma}} \\
  \end{align*}
\end{frame}


\begin{frame}{Equational theory \hfill \fbox{$\teq{\Gamma}{t}{u}{A}$}}
  \begin{multicols}{2}
    \infax[ERefl]
    {\teq{\Gamma}{t}{t}{A}}

    \infrule[ESym]
    {\teq{\Gamma}{t_1}{t_2}{A}}
    {\teq{\Gamma}{t_2}{t_1}{A}}

    \infrule[E-$\beta$]
    {\tp{\Gamma \cext \bd{x}{A}}{t}{B} \quad \tp{\Gamma}{u}{A}}
    {\teq{\Gamma}{\app{(\lam{x}{t})}{u}}{\subst{t}{\sing{x}{u}}}{B}}

    \infrule[EAbs]
    {\teq{\Gamma \cext \bd{x}{A}}{t}{t'}{B}}
    {\teq{\Gamma}{\lam{x}{t}}{\lam{x}{t'}}{\fun{A}{B}}}

    \infrule[ETrans]
    {\teq{\Gamma}{t_1}{t_2}{A} \quad \teq{\Gamma}{t_2}{t_3}{A}}
    {\teq{\Gamma}{t_1}{t_3}{A}}

    \infrule[E-$\eta$]
    {\tp{\Gamma}{t}{\fun{A}{B}} \quad \v{x} \notin \fv(t)}
    {\teq{\Gamma}{t}{\lam{x}{\app{t}{\v{x}}}}{\fun{A}{B}}}

    \infrule[EApp]
    {\teq{\Gamma}{t}{t'}{\fun{A}{B}} \quad \teq{\Gamma}{u}{u'}{A}}
    {\teq{\Gamma}{\app{t}{u}}{\app{t'}{u'}}{B}}
  \end{multicols}
\end{frame}


\section{Explicit substitutions}
\label{sec:explicit}

\begin{frame}{Explicit Substitutions}
  Substitutions are finite maps $\s{\sigma}\colon \mathcal{V} \to \mathcal{T}$
  from variables to terms.

  \bigskip

  But so far, they were meta-theoretic.

  \bigskip\pause

  Let's make them syntactic!

  \bigskip\pause

  \begin{alignat*}{2}
    \mathcal{S} \quad \ni &\quad &
    \s{\rho}, \s{\sigma}, \s{\tau} &::=
    \sempty \DefOr \pair{\sigma}{t} \DefOr \wk \DefOr \id \DefOr
      \scomp{\rho}{\sigma}
      \tag{Substitution}\\
  \end{alignat*}

\end{frame}


\begin{frame}{Update: term typing \hfill \fbox{$\tp{\Gamma}{t}{A}$}}

  \infrule[Sub]
  {\sub{\Gamma}{\sigma}{\Delta} \quad \tp{\Delta}{t}{A}}
  {\tp{\Gamma}{\subst{t}{\sigma}}{A}}

\end{frame}


\begin{frame}{Substitution typing \hfill \fbox{$\sub{\Gamma}{\sigma}{\Delta}$}}
  \infax[Empty]
  {\sub{\Gamma}{\sempty}{\cempty}}

  \infrule[Pair]
  {\sub{\Gamma}{\sigma}{\Delta} \quad \tp{\Gamma}{t}{A}}
  {\sub{\Gamma}{\pair{\sigma}{\sing{x}{t}}}{\Delta \cext \bd{x}{A}}}

  \infax[Weaken]
  {\sub{\Gamma \cext \bd{x}{A}}{\wk}{\Gamma}}

  \infax[Id]
  {\sub{\Gamma}{\id}{\Gamma}}

  \infrule[Comp]
  {\sub{\Gamma}{\sigma}{\Delta} \quad \sub{\Delta}{\rho}{E}}
  {\sub{\Gamma}{\scomp{\rho}{\sigma}}{E}}
\end{frame}


\begin{frame}{Update: term equality \hfill \fbox{$\teq{\Gamma}{t}{u}{A}$}}

  \infrule[E-$\beta$]
  {\tp{\Gamma \cext \bd{x}{A}}{t}{B} \quad \tp{\Gamma}{u}{A}}
  {\teq{\Gamma}{\app{(\lam{x}{t})}{u}}{\subst{t}{\pair{\id}{\sing{x}{u}}}}{B}}
  %
  \pause
  %
  New rules:
  %
  \bigskip

  \begin{minipage}[t]{.3\linewidth}
    \infrule[ESub]
    {\seq{\Gamma}{\sigma}{\sigma'}{\Delta} \\ \teq{\Delta}{t}{t'}{A}}
    {\teq{\Gamma}{\subst{t}{\sigma}}{\subst{t'}{\sigma'}}{A}}
  \end{minipage}\hfill
  \begin{minipage}[t]{.37\linewidth}
    \infrule[EAssoc]
    {\sub{\Gamma}{\rho}{\Delta} \\ \sub{\Delta}{\sigma}{E} \quad \tp{E}{t}{A}}
    {\teq{\Gamma}{\subst{\subst{t}{\sigma}}{\rho}}{%
        \subst{t}{\scomp{\sigma}{\rho}}}{A}}
  \end{minipage}\hfill
  \begin{minipage}[t]{.25\linewidth}
    \infrule[EId]
    {\tp{\Gamma}{t}{A}}
    {\teq{\Gamma}{\subst{t}{\id}}{t}{A}}
  \end{minipage}
  %
  \bigskip

  \begin{minipage}[t]{.4\linewidth}
    \infrule[EVarS]
    {\sub{\Gamma}{\sigma}{\Delta} \quad \tp{\Gamma}{t}{A}}
    {\teq{\Gamma}{\subst{\v{x}}{\pair{\sigma}{\sing{x}{t}}}}{t}{A}}
  \end{minipage}\hfill
  \begin{minipage}[t]{.59\linewidth}
    \infrule[EAppS]
    {\sub{\Gamma}{\sigma}{\Delta} \quad \tp{\Delta}{t}{\fun{A}{B}} \quad%
      \tp{\Delta}{u}{A}}
    {\teq{\Gamma}{\subst{(\app{t}{u})}{\sigma}}{%
        \app{\subst{t}{\sigma}}{\subst{u}{\sigma}}}{B}}
  \end{minipage}
  %
  \infrule[EAbsS]
  {\sub{\Gamma}{\sigma}{\Delta} \quad \tp{\Delta \cext \bd{x}{A}}{t}{B}}
  {\teq{\Gamma}{\subst{(\lam{x}{t})}{\sigma}}{%
      \lam{x}{\subst{t}{\pair{\scomp{\sigma}{\wk}}{\sing{x}{\v{x}}}}}}{%
      \fun{A}{B}}}
\end{frame}


\begin{frame}{Typing the RHS of $\beta$-contraction}

  \begin{prooftree}
    \AxiomC{$\tp{\Gamma \cext \bd{x}{A}}{t}{B}$}
      \AxiomC{}
      \LeftLabel{\rulelabel{Id}}
      \UnaryInfC{$\sub{\Gamma}{\id}{\Gamma}$}
        \AxiomC{$\tp{\Gamma}{u}{A}$}
      \LeftLabel{\rulelabel{Pair}}
      \BinaryInfC{$\sub{\Gamma}{\pair{\id}{\sing{x}{u}}}{%
        \Delta \cext \bd{x}{A}}$}
    \LeftLabel{\rulelabel{Sub}}
    \BinaryInfC{$\tp{\Gamma}{\subst{t}{\pair{\id}{\sing{x}{u}}}}{B}$}
  \end{prooftree}

\end{frame}


\begin{frame}{Substitution equality (1/2)%
    \hfill \fbox{$\seq{\Gamma}{\sigma}{\rho}{\Delta}$}}

  \infrule[SEAssoc]
  {\sub{\Gamma}{\rho}{\Delta} \quad
    \sub{\Delta}{\sigma}{E} \quad \sub{E}{\tau}{\Phi}}
  {\seq{\Gamma}{\scomp{(\scomp{\tau}{\sigma})}{\rho}}{%
      \scomp{\tau}{(\scomp{\sigma}{\rho})}}{\Phi}}

  \begin{minipage}[t]{.31\linewidth}
    \infrule[SEIdL]
    {\sub{\Gamma}{\sigma}{\Delta}}
    {\seq{\Gamma}{\scomp{\id}{\sigma}}{\sigma}{\Delta}}
  \end{minipage}\hfill
  \begin{minipage}[t]{.31\linewidth}
    \infrule[SEIdR]
    {\sub{\Gamma}{\sigma}{\Delta}}
    {\seq{\Gamma}{\scomp{\sigma}{\id}}{\sigma}{\Delta}}
  \end{minipage}\hfill
  \begin{minipage}[t]{.28\linewidth}
    \infax[SERefl]
    {\seq{\Gamma}{\sigma}{\sigma}{\Delta}}
  \end{minipage}
  \bigskip

  \begin{minipage}[t]{.33\linewidth}
    \infrule[SESym]
    {\seq{\Gamma}{\sigma_1}{\sigma_2}{\Delta}}
    {\seq{\Gamma}{\sigma_2}{\sigma_1}{\Delta}}
  \end{minipage}\hfill
  \begin{minipage}[t]{.62\linewidth}
    \infrule[SETrans]
    {\seq{\Gamma}{\sigma_1}{\sigma_2}{\Delta} \quad
      \seq{\Gamma}{\sigma_2}{\sigma_3}{\Delta}}
    {\seq{\Gamma}{\sigma_1}{\sigma_3}{\Delta}}
  \end{minipage}

  \infrule[SEPairS]
  {\sub{\Gamma}{\rho}{\Delta} \quad \sub{\Delta}{\sigma}{E} \quad
    \tp{\Delta}{t}{A}}
  {\seq{\Gamma}{\scomp{\pair{\sigma}{\sing{x}{t}}}{\rho}}{%
    \pair{\scomp{\sigma}{\rho}}{\sing{x}{\subst{t}{\rho}}}}{E, \bd{x}{A}}}
\end{frame}


\begin{frame}{Substitution equality (2/2)%
    \hfill \fbox{$\seq{\Gamma}{\sigma}{\rho}{\Delta}$}}

  \infrule[SE-$\beta$]
  {\sub{\Gamma}{\sigma}{\Delta} \quad \tp{\Gamma}{t}{A}}
  {\seq{\Gamma}{\scomp{\wk}{\pair{\sigma}{\sing{x}{t}}}}{\sigma}{\Delta}}

  \infax[SE-$\eta$]
  {\seq{\Gamma \cext \bd{x}{A}}{\id}{%
      \pair{\wk}{\sing{x}{\v{x}}}}{\Gamma \cext \bd{x}{A}}}

  \infrule[SEEmpty]
  {\sub{\Gamma}{\sigma}{\cempty} \quad \sub{\Gamma}{\sigma'}{\cempty}}
  {\seq{\Gamma}{\sigma}{\sigma'}{\cempty}}

  \infrule[SEComp]
  {\seq{\Gamma}{\rho}{\rho'}{\Delta} \quad
      \seq{\Delta}{\sigma}{\sigma'}{E}}
    {\seq{\Gamma}{\scomp{\sigma}{\rho}}{\scomp{\sigma'}{\rho'}}{E}}

  \infrule[SEPair]
  {\seq{\Gamma}{\sigma}{\sigma'}{\Delta} \quad \teq{\Gamma}{t}{t'}{A}}
  {\seq{\Gamma}{\pair{\sigma}{\sing{x}{t}}}{%
      \pair{\sigma}{\sing{x}{t'}}}{\Delta \cext \bd{x}{A}}}
\end{frame}


\section{De Bruijn indices}
\label{sec:debruijn}

\renewcommand{\lam}[2]{\t{\lambda #2}}
\renewcommand{\bd}[2]{#2}
\renewcommand{\sing}[2]{\t{#2}}

\begin{frame}{Getting rid of variables: de Bruijn indices}

  \begin{alignat*}{2}
    \mathcal{V} \quad \ni &\quad&
      \v{0}, \v{1}, \v{2}, &\dots
      \tag{Variable}\\
    \mathcal{T} \quad \ni &&
      \t{s}, \t{t}, \t{u} &::= \v{i} \DefOr \lam{x}{t} \DefOr
      \app{s}{t} \DefOr \subst{t}{\sigma}
      \tag{Term}\\
    \mathcal{A} \quad \ni &&
      A, B, C &::= \base \DefOr A \to B
      \tag{Type}\\
    \mathcal{C} \quad \ni &&
      \Gamma, \Delta, E &::= \cempty \DefOr \Gamma \cext \bd{x}{A}
      \tag{Context}\\
    \mathcal{S} \quad \ni &&
      \s{\rho}, \s{\sigma}, \s{\tau} &::=
      \sempty \DefOr \pair{\sigma}{t} \DefOr \wk \DefOr \id \DefOr
      \scomp{\rho}{\sigma}
      \tag{Substitution}\\
  \end{alignat*}
\end{frame}

\begin{frame}{Typing with de Bruijn indices}
  \infax[Var]
  {\tp{\bd{x_n}{A_n}, \dotsc, \bd{x_0}{A_0}}{\v{i}}{A_i}}

  \infrule[Abs]
  {\tp{\Gamma \cext \bd{x}{A}}{t}{B}}
  {\tp{\Gamma}{\lam{x}{t}}{\fun{A}{B}}}

  \infrule[Pair]
  {\sub{\Gamma}{\sigma}{\Delta} \quad \tp{\Gamma}{t}{A}}
  {\sub{\Gamma}{\pair{\sigma}{\sing{x}{t}}}{\Delta \cext \bd{x}{A}}}

  \infrule[E-$\beta$]
  {\tp{\Gamma \cext \bd{x}{A}}{t}{B} \quad \tp{\Gamma}{u}{A}}
  {\teq{\Gamma}{\app{(\lam{x}{t})}{u}}{\subst{t}{\pair{\id}{\sing{x}{u}}}}{B}}
\end{frame}

\newcommand{\suc}{\s{\operatorname{suc}}}

\begin{frame}{Getting rid of indices: use just $\vz$ + weakening}
  \bigskip
  \emph{Observation:} we can express natural numbers in Peano-style.

  \only<1>{%
  \begin{align*}
    \v{1} &= \suc(\vz)\\
    \v{2} &= \suc(\suc(\vz))\\
    \v{3} &= \suc(\suc(\suc(\vz)))\\
    &\; \; \vdots
  \end{align*}}%
  \only<2->{%
  \begin{align*}
    \v{1} &= \subst{\vz}{\wk}\\
    \v{2} &= \subst{\subst{\vz}{\wk}}{\wk}\\
    \v{3} &= \subst{\subst{\subst{\vz}{\wk}}{\wk}}{\wk}\\
    &\; \; \vdots
  \end{align*}}%
  \vspace{-2\baselineskip}%
  \pause\pause
  \begin{alignat*}{2}
    \mathcal{V} \quad \ni &\quad&
      \vz & \tag{Variable}\\
    \mathcal{T} \quad \ni &&
      \t{s}, \t{t}, \t{u} &::= \vz \DefOr \lam{x}{t} \DefOr
      \app{s}{t} \DefOr \subst{t}{\sigma}
      \tag{Term}\\
    \mathcal{A} \quad \ni &&
      A, B, C &::= \base \DefOr A \to B
      \tag{Type}\\
    \mathcal{C} \quad \ni &&
      \Gamma, \Delta, E &::= \cempty \DefOr \Gamma \cext \bd{x}{A}
      \tag{Context}\\
    \mathcal{S} \quad \ni &&
      \s{\rho}, \s{\sigma}, \s{\tau} &::=
      \sempty \DefOr \pair{\sigma}{t} \DefOr \wk \DefOr \id \DefOr
      \scomp{\rho}{\sigma}
      \tag{Substitution}
  \end{alignat*}
\end{frame}

\begin{frame}{Typing with de Bruijn indices}
  \infax[Var]
  {\tp{\Gamma \cext \bd{\vz}{A}}{\vz}{A}}

  \infrule[Abs]
  {\tp{\Gamma \cext \bd{x}{A}}{t}{B}}
  {\tp{\Gamma}{\lam{x}{t}}{\fun{A}{B}}}

  \infrule[Pair]
  {\sub{\Gamma}{\sigma}{\Delta} \quad \tp{\Gamma}{t}{A}}
  {\sub{\Gamma}{\pair{\sigma}{\sing{x}{t}}}{\Delta \cext \bd{x}{A}}}

  \infrule[E-$\beta$]
  {\tp{\Gamma \cext \bd{x}{A}}{t}{B} \quad \tp{\Gamma}{u}{A}}
  {\teq{\Gamma}{\app{(\lam{x}{t})}{u}}{\subst{t}{\pair{\id}{\sing{x}{u}}}}{B}}
\end{frame}


\section{Correspondences}
\label{sec:CHL}

\begin{frame}{Getting rid of contexts: products}

  \emph{Observation:} we can express contexts as (cartesian) products.

  \only<1>{
  \begin{alignat*}{2}
    \mathcal{V} \quad \ni &\quad&
      \vz & \tag{Variable}\\
    \mathcal{T} \quad \ni &&
      \t{s}, \t{t}, \t{u} &::= \vz \DefOr \lam{x}{t} \DefOr
      \app{s}{t} \DefOr \subst{t}{\sigma}
      \tag{Term}\\
    \mathcal{A} \quad \ni &&
      A, B, C &::= \base \DefOr A \to B
      \tag{Type}\\
    \mathcal{C} \quad \ni &&
      \Gamma, \Delta, E &::= \cempty \DefOr \Gamma \cext \bd{x}{A}
      \tag{Context}\\
    \mathcal{S} \quad \ni &&
      \s{\rho}, \s{\sigma}, \s{\tau} &::=
      \sempty \DefOr \pair{\sigma}{t} \DefOr \wk \DefOr \id \DefOr
      \scomp{\rho}{\sigma}
      \tag{Substitution}
  \end{alignat*}}
  \only<2>{
  \begin{alignat*}{2}
    \mathcal{V} \quad \ni &\quad&
      \v{\pi_2} & \tag{Variable}\\
    \mathcal{T} \quad \ni &&
      \t{s}, \t{t}, \t{u} &::= \v{\pi_2} \DefOr \lam{x}{t} \DefOr
      \app{s}{t} \DefOr \subst{t}{\sigma}
      \tag{Term}\\
    \mathcal{A} \quad \ni &&
      A, B, C &::= \base \DefOr A \to B
      \tag{Type}\\
    \mathcal{C} \quad \ni &&
      \Gamma, \Delta, E &::= \unit \DefOr \Gamma \times \bd{x}{A}
      \tag{Context}\\
    \mathcal{S} \quad \ni &&
      \s{\rho}, \s{\sigma}, \s{\tau} &::=
      \sempty \DefOr \pair{\sigma}{t} \DefOr \s{\pi_1} \DefOr \id \DefOr
      \scomp{\rho}{\sigma}
      \tag{Substitution}
  \end{alignat*}}
  \only<3>{
  \begin{alignat*}{2}
    \mathcal{V} \quad \ni &\quad&
      \v{\pi_2} & \tag{Variable}\\
    \mathcal{T} \quad \ni &&
      \t{s}, \t{t}, \t{u} &::= \v{\pi_2} \DefOr \lam{x}{t} \DefOr
      \app{s}{t} \DefOr \subst{t}{\sigma}
      \tag{Term}\\
    \mathcal{A} \quad \ni &&
    A, B, C, \Gamma, \Delta, E &::= \base \DefOr A \to B \DefOr
      \unit \DefOr \Gamma \times \bd{x}{A}
      \tag{Type}\\
    \mathcal{S} \quad \ni &&
      \s{\rho}, \s{\sigma}, \s{\tau} &::=
      \sempty \DefOr \pair{\sigma}{t} \DefOr \s{\pi_1} \DefOr \id \DefOr
      \scomp{\rho}{\sigma}
      \tag{Substitution}
  \end{alignat*}}
  \only<4->{
  \begin{alignat*}{2}
    \mathcal{V} \; \ni &\;&
      \v{\pi_2} & \tag{Variable}\\
    \mathcal{T} \; \ni &&
      \t{s}, \t{t}, \t{u}, \s{\rho}, \s{\sigma}, \s{\tau} &::=
      \v{\pi_2} \DefOr \lam{x}{t} \DefOr \app{s}{t} \DefOr
      \sempty \DefOr \pair{\sigma}{t} \DefOr \s{\pi_1} \DefOr \id \DefOr
      \scomp{\rho}{\sigma}
      \tag{Term}\\
    \mathcal{A} \; \ni &&
    A, B, C, \Gamma, \Delta, E &::= \base \DefOr A \to B \DefOr
      \unit \DefOr \Gamma \times \bd{x}{A}
      \tag{Type}
  \end{alignat*}}
\end{frame}

\newcommand{\typingrules}{%
  \infax[Var]
  {\tp{\Gamma \cext \bd{\vz}{A}}{\vz}{A}}

  \infrule[Abs]
  {\tp{\Gamma \cext \bd{x}{A}}{t}{B}}
  {\tp{\Gamma}{\lam{x}{t}}{\fun{A}{B}}}

  \infrule[App]
  {\tp{\Gamma}{t}{\fun{A}{B}} \quad \tp{\Gamma}{u}{A}}
  {\tp{\Gamma}{\app{t}{u}}{B}}

  \infax[Empty]
  {\sub{\Gamma}{\sempty}{\cempty}}

  \infrule[Pair]
  {\sub{\Gamma}{\sigma}{\Delta} \quad \tp{\Gamma}{t}{A}}
  {\sub{\Gamma}{\pair{\sigma}{\sing{x}{t}}}{\Delta \cext \bd{x}{A}}}

  \infax[Weaken]
  {\sub{\Gamma \cext \bd{x}{A}}{\pi_1}{\Gamma}}

  \infrule[Comp]
  {\sub{\Gamma}{\rho}{\Delta} \quad \sub{\Delta}{\sigma}{E}}
  {\sub{\Gamma}{\scomp{\sigma}{\rho}}{E}}

  \infax[Id]
  {\sub{\Gamma}{\id}{\Gamma}}
}

\begin{frame}{Term typing \hfill \fbox{$\tp{\Gamma}{t}{A}$}}
  \begin{multicols}{2}
    \typingrules
  \end{multicols}
\end{frame}

\renewcommand{\vz}{\v{\pi_2}}
\renewcommand{\wk}{\s{\pi_1}}
\renewcommand{\subst}[2]{\scomp{#1}{\t{#2}}}
\renewcommand{\cempty}{\unit}
\renewcommand{\cext}{\times}

\begin{frame}{Term typing \hfill \fbox{$\tp{\Gamma}{t}{A}$}}
  \begin{multicols}{2}
    \typingrules
  \end{multicols}
\end{frame}

\renewcommand{\tp}[3]{\t{#2}\colon #1 \longrightarrow #3}
\renewcommand{\sub}[3]{\s{#2}\colon #1 \longrightarrow #3}

\begin{frame}{Combinator typing \hfill \fbox{$\tp{\Gamma}{t}{A}$}}
  \begin{multicols}{2}
    \typingrules
  \end{multicols}
\end{frame}

\renewcommand{\cempty}{\top}
\renewcommand{\c}{\wedge}
\renewcommand{\fun}[2]{#1 \supset #2}

\renewcommand{\tp}[3]{#1 \Longrightarrow #3}
\renewcommand{\sub}[3]{#1 \Longrightarrow #3}

\begin{frame}{Combinatory IPL \hfill \fbox{$\tp{\Gamma}{t}{A}$}}
  \begin{multicols}{2}
    \typingrules
  \end{multicols}
\end{frame}



%%% Backup slides  %%%

% Save the frame number of the previous slide to restore the frame
% counter after the backup slides.
%\begin{backupslides}

\appendix

%\end{backupslides}

%%% Bibliography

%\bibliography{biblio}
%\bibliographystyle{apalike}

\begin{frame}[noframenumbering]{}
  \vspace{1cm}
  \begin{center}
    \includegraphics[height=2cm]{img/cc_by_bw}
    \vspace{2em}

    Except where otherwise noted, this work is licensed under
    \vspace{1em}

    \url{http://creativecommons.org/licenses/by/3.0/}
  \end{center}
\end{frame}

\end{document}
