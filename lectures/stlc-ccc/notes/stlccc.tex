\documentclass[a4paper,fleqn]{scrartcl}

\usepackage{amsmath}
\usepackage{amsfonts}
\usepackage[style=authoryear]{biblatex}
\usepackage[all]{xy}

\addbibresource{stlccc.bib}

\title{From the simply-typed lambda-calculus to
  cartesian closed categories}
\author{Andreas Abel}
\date{December 2017}

\newcommand{\wttm}{\ensuremath{\mathsf{Tm}\,\Gamma\,a}}
\newcommand{\Set}{\mathsf{Set}}
\newcommand{\Ob}{\mathsf{Ob}}
\newcommand{\Hom}{\mathsf{Hom}}
\newcommand{\tid}{\mathsf{id}}
\newcommand{\comp}{\circ}
\newcommand{\To}{\Rightarrow}
\newcommand{\too}{\longrightarrow}
\newcommand{\tcurry}{\mathsf{curry}}
\newcommand{\tapply}{\mathsf{apply}}
\newcommand{\teval}{\mathsf{eval}}

\newcommand{\der}{\,\vdash}
\newcommand{\subst}[3]{#3[#2{\leftarrow}#1]}
\newcommand{\substp}[3]{(#3)[#2{\leftarrow}#1]}
\newcommand{\ru}[2]{\dfrac{#1}{#2}}
\newcommand{\nru}[3]{#1\ \ru{#2}{#3}}
\newcommand{\nrux}[4]{\nru{#1}{#2}{#3}\ #4}
\newcommand{\rux}[3]{\ru{#1}{#2}\ #3}

\newcommand{\Var}{\mathsf{Var}}
\newcommand{\Ty}{\mathsf{Ty}}
\newcommand{\Tm}{\mathsf{Tm}}
\newcommand{\Exp}{\mathsf{Exp}}
\newcommand{\Cxt}{\mathsf{Cxt}}
\newcommand{\Subst}{\mathsf{Subst}}

\newcommand{\GG}{\ensuremath{\Gamma}}
\newcommand{\GD}{\ensuremath{\Delta}}
\newcommand{\Ge}{\ensuremath{\varepsilon}}
\newcommand{\Gs}{\ensuremath{\sigma}}


\begin{document}
\maketitle

Outline of the transformation.
\begin{enumerate}
\item Simply-typed lambda-calculus (named variables).
\item Explicit substitutions and judgemental equality.
\item Nameless (de Bruijn) presentation.
\item Removing the judgement for well-typed variables: $0$ is the only
  variable, the others are represented using the weakening
  substitution.
\item Conflate contexts and types, substitutions and terms: we arrive
  at an internal language for Cartesian closed categories.
\end{enumerate}

\section{From STLC to CCCs}

\subsection{STLC with named variables}

Syntax.
\[
\begin{array}{lllrl@{\qquad}l}
\Var & \ni & x & & & \mbox{variable, e.g. string} \\
\Tm & \ni & t,u & ::= & x \mid \lambda x.\,t \mid t \, u & \mbox{term}
  \\
\Ty  & \ni & A,B,C & ::= & o \mid A \To B & \mbox{simple type} \\
\Cxt & \ni & \Gamma, \Delta & ::= & \varepsilon \mid \Gamma.x{:}A
  & \mbox{typing context}\\
\end{array}
\]
Judgements.
\[
\begin{array}{l@{\qquad}l}
  \Gamma \der t : A & \mbox{in context $\Gamma$, term $t$ has type $A$ } \\
  \Gamma \der t = t' : A & \mbox{in context $\Gamma$, term $t$ is equal to $t'$ of type $A$ } \\
\end{array}
\]
Equality rules.
\[
  \nru{\beta
    }{\Gamma.x{:}A \der t : B \qquad \Gamma \der u : A
    }{\Gamma \der (\lambda x.\,t)\,u = \subst u x t : B
    }
\qquad
  \nrux{\eta
   }{\Gamma \der t : A \To B
   }{\Gamma \der t = \lambda x.\,t\,x
   }{x\#t}
\]
Substitution $\subst u x t$ is defined by recursion on $t$.
We write $z\#M$ when $z$ does not clash with the free variables of
terms contained in $M$.
\[
\begin{array}{rll@{\qquad}l}
  \subst u x x & = & u \\
  \subst u x y & = & y & \mbox{if } x \not= y \\
  \subst u x {(t_1\,t_2)} & = & (\subst u x {t_1})\, (\subst u x {t_2}) \\
  \subst u x {(\lambda x.\,t)} & = & \lambda x.\,t \\
  \subst u x {(\lambda y.\,t)} & = & \lambda y.\,{\subst u x t}
    & \mbox{if } y\#u \\
  \subst u x {(\lambda y.\,t)} & = & \lambda z.\,{\subst u x {\subst z y t}}
    & \mbox{for some } z\#(u,y) \\
\end{array}
\]
The complication in the last case is necessary to avoid capture of
free variables of $u$, e.g., consider $\substp y x {\lambda y.\,x}$.
The result should be $\lambda z.\,y$ for some $z \not= y$, not
$\lambda y.\,y$.
Note that because of the last case, substitution is not formally
structurally recursive.  However, it could be defined by recursion on
the height or size of term $t$, since this measure does not change by
a renaming $\subst z y t$.

\subsection{Parallel substitution}

Structural recursiveness of substitution can be restored if we substitute several variables at once, or even \emph{all} free variables.

\noindent
Syntax.
\[
\begin{array}{lllrl@{\qquad}l}
\Subst & \ni & \sigma & ::= & \varepsilon \mid \sigma,x{\mapsto}u
  & \mbox{substitution} \\
\end{array}
\]
Application of a substitution $t\,\sigma$ is defined by recursion on $t$.
\[
\begin{array}{rll@{\qquad}l}
  x\,\sigma & = & \sigma(x) \\
  (t_1\,t_2)\,\sigma & = & (t_1\,\sigma)\,(t_2\,\sigma) \\
  (\lambda y.\,t)\,\sigma & = & \lambda z.\,t\,(\sigma, y{\mapsto}z)
    & \mbox{for some } z \# \sigma \\
\end{array}
\]
Substitution typing $\Gamma \der \sigma : \Delta$.
\[
  \ru{}{\Gamma \vdash \Ge : \Ge}
\qquad
  \ru{\Gamma \vdash \Gs : \GD \qquad \Gamma \vdash u : A
     }{\Gamma \vdash (\Gs,x{\mapsto}u) : (\GD.x{:}A)
     }
\]
Exercise: Prove the substitution lemma:
If $\Delta \der t : A$ and $\Gamma \der \Gs : \Delta$
then $\Gamma \der t\,\Gs : A$.

\subsection{The category of substitutions}

Identity substitution $\tid_\Gamma$ is defined by recursion on $\Gamma$ and substitution composition $\sigma\comp\sigma'$ by recursion on $\sigma$, applying $\sigma'$ to every term in $\sigma$.
\[
  \ru{}{\Gamma \der \tid_\Gamma : \Gamma}
  \qquad
  \ru{\Gamma_2 \der \sigma : \Gamma_3 \qquad
      \Gamma_1 \der \sigma' : \Gamma_2
    }{\Gamma_1 \der \sigma\comp\sigma' : \Gamma_3}
\]
Exercise: Prove the category laws (identity and associativity)!

The $\beta$-law is now formulated as:
\[
  \nru{\beta
    }{\Gamma.x{:}A \der t : B \qquad \Gamma \der u : A
    }{\Gamma \der (\lambda x.\,t)\,u = t\,(\tid_\Gamma,x{\mapsto}u) : B
    }
\]

\subsection{Explicit substitutions}

We can ``freeze'' substitution by making it part of the term grammar,
and animate it via equality rules.
\[
\begin{array}{lllrl@{\qquad}l}
\Tm & \ni & t,u & ::= & \dots \mid t \comp \Gs
  & \mbox{explicit substitution} \\
\end{array}
\]
All the computation rules for substitutions become equality rules.
\[
\rux{\Delta.x{:}A \der t : B \qquad \Gamma \der \Gs : \Delta
   }{\Gamma \der (\lambda x.\,t) \comp \Gs
              = \lambda y.\,t \comp (\Gs,x{\mapsto}y) : B
   }{y \# \Gamma}
\]
Exercise: Write down the complete set of equality rules!

\subsection{Nameless terms}

Handling of names is a constant hassle.  Semantically, a variable is
not more than a pointer to a binding site, so let us implement this,
following \textcite{Bruijn72}!

We drop names from the syntax everywhere, and replace variables by
natural numbers denoting how many binders to skip when traversing
towards the root to reach the binding site.  E.g. the named term
$\lambda x.\,x\,(\lambda y.\,x\,(\lambda x.\,x\,y))$ becomes
$\lambda.\,0\,(\lambda.\,1\,(\lambda.\,0\,1))$.

%\noindent
Syntax.
\[
\begin{array}{lllrl@{\qquad}l}
\Var & \ni & x & ::= & 0 \mid 1 + x & \mbox{de Bruijn index} \\
\Tm & \ni & t,u & ::= & x \mid \lambda t \mid t \, u \mid t \comp \Gs & \mbox{term}
  \\
\Cxt & \ni & \Gamma,\Delta & ::= & \varepsilon \mid \Gamma.A
  & \mbox{typing context}\\
\Subst & \ni & \sigma & ::= & \varepsilon \mid \sigma,u
  & \mbox{substitution} \\
\end{array}
\]
The new judgement $\Gamma \ni x : A$ asserts that index $x$ points to an occurrence of $A$ in the list $\Gamma$:
\[
  \ru{}{\Gamma.A \ni 0 : A}
\qquad
  \ru{\Gamma \ni x : A}{\Gamma.B \ni 1 + x : A}
\]
Typing $\Gamma \der t : A$ of terms becomes:
\[
  \ru{\Gamma \ni x : A}{\Gamma \der x : A}
\qquad
  \ru{\Gamma.A \der t : B
    }{\Gamma \der \lambda t : A \To B}
% \qquad
%   \ru{\Gamma \der t : A \To B \qquad
%       \Gamma \der u : A
%     }{\Gamma \der t\,u : B}
\]
Exercise: What becomes of the rules for application and explicit substitution?

Substitution typing $\Gamma \der \sigma : \Delta$ becomes:
\[
  \ru{}{\Gamma \vdash \Ge : \Ge}
\qquad
  \ru{\Gamma \vdash \Gs : \GD \qquad \Gamma \vdash u : A
     }{\Gamma \vdash (\Gs,u) : (\GD.A)
     }
\]
Exercise: Write out the equality rules!

\subsection{Explicit weakening}

Typings stay valid under context extensions: In the named case, $\Gamma \der t : A$ implies $\Gamma.x{:}B \der t : A$ in case $x \# \Gamma$.  In the nameless case, this can be realized by a substitution (exercise!), but may also become a constructor for terms.  As a consequence, we only need variable $0$, and can drop the syntactic class $\Var$ entirely!

Syntax.
\[
\begin{array}{lllrl@{\qquad}l}
\Tm & \ni & t,u & ::= & 0 \mid 1{+}\;t \mid \dots & \mbox{term} \\
\end{array}
\]
Typing.
\[
  \ru{}{\Gamma.A \der 0 : A}
\qquad
  \ru{\Gamma \der t : A}{\Gamma.B \der {1+}\; t : A}
\]
Exercise: Equality rules!

\subsection{CCCs: Contexts as products, substitutions as tuples}

If we view contexts $\Ge.A_1.\cdots.A_n$ simply as products
$((1 \times A_1) \times \cdots) \times A_n$ with the unit type $1$
replacing the empty context, substitutions $\Gs$ become simply nested
tuples!  The typing judgements become morphism typing:
\[
\begin{array}{lcl}
  \Gamma \der t : A & \leadsto & t : \Gamma \too A \\
  \Gamma \der \Gs : \Delta & \leadsto & \Gs : \Gamma \too \Delta \\
\end{array}
\]
Substitution typing:
\begin{gather*}
  \ru{}{! : A \to 1}
\qquad
  \ru{t : A \to B \qquad u : A \to C
    }{(t,u) : A \to B \times C}
\\[2ex]
  \ru{}{\tid_A : A \too A}
\qquad
  \ru{t : B \too C \qquad u : A \too B}{t \comp u : A \too C}
\end{gather*}
Variable $0$ becomes the second projection $\pi_2$ and weakening $1{+}\;t$ a composition $t \comp \pi_1$ with the first projection.
\[
  \ru{}{\pi_2 : A \times B \too B}
\qquad
  \ru{}{\pi_1 : A \times B \too A}
\]
Abstraction $\lambda t$ is called \emph{currying} and application $t\,u$ is decomposed into $\teval \comp (t,u)$.
\[
  \ru{t : A \times B \too C}{\lambda t : A \too (B \To C)}
\qquad
  \ru{}{\teval : (A \To B) \times A \too B}
\]
Exercise: Write out the equational theory of CCCs.

Raw syntax of the initial CCC:
\[
\begin{array}{lllrl@{\qquad}l}
\Ob & \ni & A,B,C & ::= & 1 \mid A \times B \mid A \To B \\
\Tm & \ni & t,u   & ::= & \tid \mid t \comp u \mid {!} \mid (t,u) \mid \pi_1 \mid \pi_2 \mid \lambda t \mid \teval \\
\end{array}
\]

\clearpage

\section{Exercises (Pen and Paper)}

\subsection{Simply-typed lambda-calculus with explicit substitutions}

\begin{enumerate}
\item Write down the typing rules for simply-typed lambda-calculus.
\item Write down the typing rules for substitutions.
\item Write down the equality rules.
  \begin{enumerate}
  \item $\beta$- and $\eta$-equality.
  \item Rules for the propagation of explicit substitutions into
    terms.
  \item Rules for the composition of substitutions.
  \end{enumerate}
\end{enumerate}

\subsection{Categories}

\begin{enumerate}

\item Category of monoids.
  \begin{enumerate}
  \item Define the category of (small) monoids (where the carrier of
    the monoid is a set).
  \item Show that the length function for lists is a monoid morphism.
  \end{enumerate}

\item Category of categories.
  \begin{enumerate}
  \item Generalize the notion of monoid morphism to the concept of
    morphism between categories.
  \item Show that the (small) categories (where the objects form a
    set) form a category themselves.  (This category is large in the
    sense that objects form a class.)
  \end{enumerate}

\end{enumerate}

\subsection{Products}

\newcommand{\C}{\mathcal{C}}

\begin{enumerate}
\item Define the product in the category of monoids.
\item Let $1$ denote the terminal object of some category $\C$.
  Show that the following span is a product of $A$ and $1$.
\[
\xymatrix{
  A & A \ar[l]_{\tid} \ar[r]^{!} & 1
}
\]
\end{enumerate}

\subsection{Cartesian closed categories}

Recall $\teval : (A \To B) \times A \too B$ and
$\tcurry\, f : C \too (A \To B)$ for $f : C \times A \too B$.
\begin{enumerate}
\item Show $\tcurry\,\teval = \tid$.
\item Show $A \cong (1 \To A)$.
\end{enumerate}

\clearpage

\section{Exercises (Agda)}

\subsection{Simply-typed lambda-calculus in Agda}

\begin{enumerate}
\item Represent simply-typed terms in Agda, using de Bruijn indices.
  \begin{enumerate}
  \item Code simple types $a$ and contexts $\Gamma$ as data types.
  \item Define well-typed variables as indexed data type
    $\mathsf{Var}\,\Gamma\,a$.
  \item Define well-typed terms as indexed data type \wttm.
  \end{enumerate}
\item Add explicit substitutions via an indexed data type
  $\mathsf{Sub}\,\Gamma\,\Delta$.
\item Define an equality judgement as relation between well-typed
  terms: ${\_{\cong}\_} : \wttm \to \wttm \to \mathsf{Set}$.
  Each rule is one constructor of this indexed data type.
\item Likewise, implement an equality judgement for well-typed substitutions.
\end{enumerate}

\subsection{CCCs in Agda}
\begin{enumerate}
\item An E-category is a category with an equivalence relation on
  homsets.  Define the notion of E-category in Agda.
  \begin{enumerate}
  \item There is a $\Set$ of objects $\Ob$.
  \item For each two objects $a,b : \Ob$
    there is a $\Set$ of (homo)morphisms $\Hom\,a\,b$ from $a$ to $b$.
  \item There is an equivalence relation on $\Hom\,a\,b$ (for each
    $a,b : \Ob$).
  \item There is an associative morphism composition $f \circ g : \Hom\,a\,c$ for each
    $f : \Hom\,b\,c$ and $g : \Hom\,a\,b$.
  \item Composition respects equality.
  \item There are morphisms $\tid\,a$ for each $a : \Ob$ which are
    left and right units for composition.
  \end{enumerate}
\item Add products and terminal objects.
\item Add exponentials.
\end{enumerate}

\subsection{Interpretation of STLC in CCCs}
\begin{enumerate}
\item Fix an arbitrary CCC.
\item Write an interpretation of types and contexts as objects in the CCC.
\item Write an interpretation of well-typed terms and substitutions as
  morphisms in the CCC.
\item Write an interpretation of judgemental equality as equality of
  morphisms in the CCC.
  \begin{itemize}
  \item First, prove each rule of judgemental equality as a theorem
    about morphisms in a CCC.
  \item Then, map the rules to these theorems.
  \end{itemize}
\end{enumerate}

\clearpage

\printbibliography

\end{document}
