\documentclass[a4paper,fleqn]{scrartcl}

% Packages
\usepackage{amsmath}
\usepackage{amsfonts}
\usepackage{amsthm}
\usepackage[style=authoryear]{biblatex}
\usepackage[inline]{enumitem}
\usepackage{xspace}
\usepackage[all]{xy}

\usepackage{hyperref} % before cleveref
\usepackage{cleveref} % wants to be last, but before the theorem defs

\newtheorem{theorem}{Theorem}
\newtheorem{lemma}{Lemma}

\theoremstyle{definition}
\newtheorem{definition}{Definition}

%\theoremstyle{remark}
\newtheorem{remark}{Remark}
\newtheorem{example}{Example}
\newtheorem*{solution}{Solution}
\newtheorem{exercise}{Exercise}

% \crefname{exercise}{exercise}{exercises}

\addbibresource{stlccc.bib}

\title{Cartesian closed categories}
\author{Andreas Abel}
\date{November 2020%
     }

% General macros
\usepackage{aamacros}

% Specific macros
\newcommand{\wttm}{\ensuremath{\mathsf{Tm}\,\Gamma\,a}}
\newcommand{\Var}{\mathsf{Var}}
\newcommand{\Ty}{\mathsf{Ty}}
\newcommand{\Tm}{\mathsf{Tm}}
\newcommand{\Exp}{\mathsf{Exp}}
\newcommand{\Cxt}{\mathsf{Cxt}}
\newcommand{\Subst}{\mathsf{Subst}}

% Categories
\let\C\relax % already defined in puenc.def (part of hyperref)
\newcommand{\C}{\mathcal{C}}
\newcommand{\Eq}{\mathsf{Eq}}



\begin{document}
\maketitle

Overview:
\begin{enumerate}
\item Categories
\item Functors and natural transformations
\item Products
\item Exponentials
\end{enumerate}

\section{Categories}
\label{sec:cat}

\subsection{Definition and examples}

\begin{definition}[Category]
  \label{def:cat}
  A category $\C$ is given by the following data:
  \begin{enumerate}
  \item Types:
    \begin{enumerate}
    \item A type $\Ob$ of \emph{objects}.
    \item For each pair of objects $A,B : \Ob$, a type $\Hom(A,B)$ of
      (homo)morphisms $f : A \too B$.
    \item For each pair of objects $A,B : \Ob$, an equivalence
      relation $\Eq(A,B)$ on $\Hom(A,B)$.
      Given $f,g : \Hom(A,B)$, we write $f = g$ for $\Eq(A,B)(f,g)$.
    \end{enumerate}
  \item Operations:
    \begin{enumerate}
    \item For each object $A : \Ob$ an automorphism $\id[A] : A \too
      A$ (identity).
    \item For each pair $f : A \too B$ and $g : B \too C$ of morphisms a
      morphism $g \comp f : A \too C$ (composition).
    \end{enumerate}
  \item Laws:
    \begin{enumerate}
    \item For each morphism $f : A \too B$ we have $\id[B] \comp f = f$
      (left identity)
      and $f \comp \id[A] = f$ (right identity).
    \item For all morphisms $f : A \too B$ and $g : B \too C$ and $h : B
      \too C$ we have $(h \comp g) \comp f = h \comp (g \comp f)$ (associativity).
    \item For all morphisms $f$, $f' : A \too B$ such that $f = f'$
      and $g$, $g' : B \too C$ such that $g = g'$ we have
      $g \comp f = g' \comp f'$ (congruence).
    \end{enumerate}
  \end{enumerate}
\end{definition}
The arrow $A \too B$ is just a nice notation for $\Hom(A,B)$.
It is also common to write $\C(A,B)$ to clarify that we mean the type $\Hom[\C](A,B)$
of morphisms of category $\C$.  Also $A : \C$ is short for $A : \Ob[\C]$.
\begin{remark}[Homsetoid]
Since a type with an equivalence relation is called a \emph{setoid} we
could just ask for a family $\Hom : \Ob \to \Ob \to \Setoid$.
\end{remark}
The prime example for categories are collections of algebraic
structures and their structure-preserving homomorphisms.
\begin{example}[Groups]
  $\Grp$ is the category of groups and group homomorphisms.  More
  precisely, the objects of $\Grp$ are groups, and an element
  $f : \Grp(A,B)$ is a function $f : A \to B$ mapping the unit of $A$
  to the unit of $B$ and the $A$-composition of two elements of $A$ to
  the $B$-composition of their images under $f$.

  Less abstractly, a group morphism $f : (A,0,+,-) \too (B,1,\times,{}^{-1})$
  has to satisfy $f(0) = 1$ and $f(a+a') = f(b) \times f(b')$.
\end{example}
\begin{exercise}[Groups] \bla
  \begin{enumerate}
  \item
  Give an example for a group morphism $f$.
  \item
  Show that a group morphism automatically preserves inverses, \ie,
  $f(-a) = (f(a))^{-1}$.
  \end{enumerate}
\end{exercise}
Analogously to groups, other algebraic structures can be organized as
categories as well (monoids, rings, fields).  We exhibit the most
basic examples:
\begin{example}[Sets]
  $\Set$ is the category of types $A$ and functions $f : A \to B$.
\end{example}
\begin{example}[Setoids]
  $\Setoid$ is the category of setoids $(A,\approx_A)$ and
  $\approx$-preserving functions, \ie, $f : A \too B$ must satisfy
  $f(a) \approx_B f(a')$ whenever $a \approx_A a'$.
\end{example}

Besides organizing algebraic structures, categories can also
\emph{implement} structures.
\begin{example}[Monoid]
  Each monoid $(M,e,\cdot)$ can be presented as category $\C_M$ with a
  single object $1$ and $\Hom(1,1)=M$.  Then $\id[1] = e$ and $f \comp
  g = f \cdot g$.
\end{example}
\begin{exercise}[Partial monoid]
  \label{ex:pmon}
  Can any partial monoid be represented as category as well?  If yes,
  how?  If no, give a counterexample!

  Recall that a set $M$ with a distinguished element $e : M$ and a
  \emph{partial} binary operation $\_\circ\_ : M \times M \rightarrow M$
  is called a \emph{partial} monoid if
  %
  \begin{enumerate*}
  \item $e \circ x = x = x \circ e$ for each element $x : M$, and
  \item $(x \times y) \times z = x \times (y \times z)$ if
    $y \times z$ and $x \times (y \times z)$, or $x \times y$ and
    $(x \times y) \times z$ are defined.
  \end{enumerate*}
\end{exercise}
\begin{example}[Preorder]
  Any preorder $(A,\leq)$ can be presented as a thin category with $\Ob =
  A$ and $\Hom(a,b) = \{ 0 \mid a \leq b \}$.  Identity is reflexivity
  and composition is transitivity.

  A category $\C$ is called \emph{thin} if each homset $\Hom[\C](A,B)$
  has \emph{at most one} inhabitant, that is for all pairs of parallel
  morphisms $f$, $f' : A \too B$ in $\C$ we have $f = f'$.
\end{example}
\begin{example}[Relations]
  The category $\Rel$ has types as objects and binary relations as
  morphisms: $\Rel(A,B) = \Pot(A \times B)$.
\end{example}
\begin{example}[Contexts and substitutions]
  Take the typing contexts $\Gamma$ of simply-typed lambda-calculus
  as objects, $\Ob = \Cxt$, and the set of substitutions
  $\Sub\,\GG\,\GD$ as morphisms from $\Gamma$ to $\Delta$.
\end{example}

\begin{definition}[Subcategory]
  \label{def:subcat}
  A category $\D$ is a \emph{subcategory} of $\C$ if $\Ob[\D]
  \subseteq \Ob[\C]$, $\Hom[\D](A,B) \subseteq \Hom[\C](A,B)$ for
  all $A,B : \Ob[\D]$, $\id[\D,A] = \id[\C,A]$ for all $A : \Ob[\D]$,
  and $g \comp_{\D} f = g \comp_{\C} f$ for all $f : \Hom[\D](A,B)$
  and $g : \Hom[\D](B,C)$.

  If $\Ob[\D] = \Ob[\C]$, the subcategory is \emph{wide}.

  If $\Hom[\D](A,B) = \Hom[\C](A,B)$ for
  all $A,B : \Ob[\D]$, the subcategory is \emph{full}.
\end{definition}
In other words, a subcategory $\D$ of $\C$ is a selection of objects
and morphisms from $\C$ that still forms a category, \ie, is closed
under identity and composition.
% \begin{remark}
%   Type-theoretically, one should rather define a subcategory of $\C$
%   by restriction predicates $P : \Ob \to$
% \end{remark}

\subsection{On the equality of objects}

Our definition of category does not include an equivalence relation on
$\Ob$.  This is by intention, speaking about object equality is not
considered pure category-theoretic spirit.  All category-theoretic
notions should respect isomorphic objects.
\begin{definition}[Isomorphism]
  An \emph{isomorphism} (short \emph{iso}) between two objects $A$ and
  $B$ is a pair of morphisms $f : A \too B$ and $g : B \too A$ such
  that $g \comp f = \id[A]$ and $f \comp g = \id[B]$.
  The existence of an isomorphism is written $A \cong B$, and the set
  of isomorphisms is denoted by $\Iso(A,B)$.
\end{definition}
\begin{lemma}[Inverse]
  Fixing $f$, the inverse $g$ is uniquely determined and denoted by $f^{-1}$.

  In other words, being an isomorphism is not a property of a pair of
  morphisms but a single morphism, that is for each pair of objects
  $A$ and $B$ the function
  %
  \begin{gather*}
    \Iso(A,B) \rightarrow \SetBuilder{ f \in \Hom(A,B) \given \exists g.\,g \comp f = \id \wedge f \comp g = \id } \quad (f,g) \mapsto f
  \end{gather*}
  %
  is a bijection of sets with inverse $f \mapsto (f,f^{-1})$.
\end{lemma}
\begin{exercise}
  Prove this!
\end{exercise}
\begin{exercise}[Subcategory of isomorphisms]
  Show that the isomorphisms of a category constitute a wide subcategory.
\end{exercise}
\begin{exercise}
  Does the concept \emph{subcategory} (\Cref{def:subcat}) honor the
  ideal that no category-theoretic concept should distinguish between
  isomorphic objects?

  If not, suggest a modification of the definition, or defend the
  current definition against the ideal.
\end{exercise}

\subsection{Operations on categories}

Some operations on the object types can be lifted to categories.
\begin{enumerate}
\item The product $\C \times \D$ of two categories forms again a
  category with $\Ob[\C \times \D] = \Ob[\C] \times \Ob[\D]$.
\item The latter can be generalized to nullary, finite, and even
  infinite products.
\item Any type can be turned into a \emph{discrete} category where the
  identities are the only morphisms.
\end{enumerate}

\begin{definition}[Opposite category]
  Given a category $\C$, its \emph{opposite} $\op\C$ has the same
  objects but flipped morphisms, $\op\C(A,B) = \C(B,A)$, and thus
  flipped composition: $f \comp_{\op\C} g = g \comp_\C f$.
\end{definition}
\begin{remark}
  The opposite category is really just the original category with
  morphisms relabeled so that source and target are formally exchanged.
\end{remark}
\begin{exercise}
  Show that $\op\C$ is indeed a category.

  Show that $\op{(\op\C)} = \C$.
\end{exercise}

\section{Functors and Natural Transformations}

A \emph{functor} $F : [\C,\D]$ is a category morphism:
\begin{definition}[Functor]
  \label{def:functor}
  Given categories $\C$ and $\D$ a functor $F : [\C,\D]$
  is given by the following data:
  \begin{enumerate}
  \item Maps:
    \begin{enumerate}
    \item A function $F_0 : \Ob[\C] \to \Ob[\D]$.
    \item For any pair of objects $A,B : \C$, a function $F_1 :
      \Hom[\C](A,B) \to \Hom[\D](F_0A,F_0B)$.
    \end{enumerate}
  \item Laws:
    \begin{enumerate}
    \item For any object $A : \C$ we have $F_1(\id[A]) = \id[F_0A]$.
    \item For any pair of morphisms $f : \C(A,B)$ and $g : \C(B,C)$ we
      have $F_1(g \comp_\C f) = F_1g \comp_\D F_1f$.
    \item For any pair of parallel morphisms $f$, $f' : \C(A,B)$ such
      that $f = f'$ we have $F_1(f) = F_1(f')$.
    \end{enumerate}
  \end{enumerate}
\end{definition}
It is common to drop the indices $0$ and $1$ and simply write, \eg,
$Ff : FA \too FB$.
Also, since there is little chance of confusion, one often writes $F :
\C \to \D$ instead of $F : [\C,\D]$.
\begin{example}[Forgetful functor]
  ``Forgetting'' algebraic structure gives rise to trivial functors,
  the so-called \emph{forgetful functors}, often denoted by $U$.  For
  example, $U : \Grp \to \Set$ maps groups to their carriers, and
  group morphisms to their underlying functions on the carriers.

  A forgetful functor does nothing to the ``values'', only changes
  their ``types''.
\end{example}
\begin{exercise}
  Define the duplication functor $\Dup : [\C,\C \times \C]$
  from a category to its square.
\end{exercise}

Since functors are not mathematical structures (such as groups and
categories) it is not obvious what the notion of morphism between two
functors $F,G : [\C,\D]$ should be.  The definition states that it is
a family of morphisms $F A \too G A$ parametric in $A$:
\begin{definition}[Natural transformation]
  Given functors $F,G : [\C,\D]$, a \emph{natural transformation}
  $\eta : F \todot G$ is a family of morphisms $\eta_A : F A \too G A$
  indexed by $A : \C$ such that for all $f : A \too B$ we have $Gf
  \comp \eta_A = \eta_B \comp Ff$.
\end{definition}
Diagrammatically, the commutation law can be depicted as follows:
\[
\xymatrix@R=8ex@C=10ex{
A \ar[d]^f & F A \ar[d]^{F f} \ar[r]^{\eta_A} & G A \ar[d]^{G f} \\
B & F B \ar[r]^{\eta_B} & G B \\
}
\]
\begin{exercise}[Functor category]
  Show that functors in $[\C,\D]$ form a category with natural
  transformations as morphisms.
\end{exercise}
\begin{definition}[$\Cat$]
  Taking categories $\C$ as objects themselves and functor sets
  $[\C,\D]$ as homsets, we arrive at the category $\Cat$ of
  categories!

  For consistency reasons $\Ob[\Cat]$ needs to be a large type
  containing categories $\C$ whose $\Ob[\C]$ is a small type.
\end{definition}
\begin{exercise}
  Prove that functors are indeed closed under composition and that
  $\Cat$ is indeed a category.
\end{exercise}
\begin{remark}[2-categories]
  In $\Cat$, the functor types $[\C,\D]$ are only taken as sets, but
  they are categories themselves!  Categories whose homsets are
  categories again are called \emph{2-categories} or
  \emph{bicategories}.  These have extra structure---we'll not dive
  further into this now.
\end{remark}


\section{Cartesian Categories}

Category theory rarely studies pure categories, but usually categories with
extra structure.

\begin{definition}[Product]
  Given $A,B : \C$, a \emph{product} of $A$ and $B$ is given by the
  following data:
  \begin{enumerate}
  \item An object $P : \C$, and
  \item a pair of morphisms $\pi_1 : P \too A$ and $\pi_2 : P \too B$,
    such that
  \item for each object $C$ and morphisms $f : C \too A$ and $g : C
    \too B$ there is a unique morphism $h : C \too P$ such that $\pi_1
    \comp h = f$ and $\pi_2 \comp h = g$.
  \end{enumerate}
  The uniqueness of $h$ justifies the notation $h = \langle f,g \rangle$.
  Since $P$ is unique up to isomorphism (see below),
  the notation $P = A \times B$
  is justified.
\end{definition}
The so-called \emph{universal property} that defines the product can
be diagrammatically displayed as follows:
\[
\xymatrix@C=8ex@R=16ex{
& \ar[dl]_f C \ar@{.>}[d]^{!h} \ar[dr]^g& \\
A & \ar[l]_{\pi_1} A \times B  \ar[r]^{\pi_2} & B \\
}
\]
%%We may also say that $P$ is a product of $A$ and $B$
\begin{example}\bla
  \begin{enumerate}
  \item The cartesian product is the product in $\Set$, $\Setoid$,
    $\Grp$ etc.
  \item In $\Sub$, the cartesian product is context concatenation.
  \end{enumerate}
\end{example}
\begin{exercise}
  What is a product in a preorder?
  Under which conditions do preorders have all products?
\end{exercise}
\begin{exercise}[Uniqueness of product]
  Let $(P,\pi_1,\pi_2)$ and $(Q,q_1,q_2)$ be both products of $A$ and
  $B$.  Show that $P \cong Q$.
\end{exercise}
\begin{exercise}[Commutativity]
  Show that $A \times B \cong B \times A$.
\end{exercise}
\begin{exercise}[Derived laws]
  Proof the following theorems using the universal property:
  \begin{enumerate}
  \item $\langle \pi_1,\pi_2 \rangle = \tid$.
  \item $\langle f,g \rangle \comp h = \langle f \comp h,\; g \comp h \rangle$.
  \end{enumerate}
\end{exercise}
\begin{exercise}[Morphism product]
  \label{ex:hom-product}
  Given $f_1 : A_1 \too B_1$ and $f_2 : A_2 \too B_2$, define $f_1
  \times f_2 : A_1 \times A_2 \too B_1 \times B_2$.
\end{exercise}

The nullary product is called the terminal object.
\begin{definition}[Terminal object]
  An object $T : \C$ is \emph{terminal} if for any object $C$ there is
  a unique morphism $h : C \too T$.

  The uniqueness of $h$ justifies the notation $h = \bang[C]$.
  Since $T$ is unique up to isomorphism (see below), it is usually
  denoted by $1$.
\end{definition}
\begin{exercise}
  Give, if it exists, the terminal object in the categories $\Set$,
  $\Setoid$, $\Grp$, $\Rel$.
\end{exercise}
\begin{exercise}
  What is a terminal object in a preorder?
\end{exercise}
\begin{exercise}
  The terminal object is unique up to isomorphism.
\end{exercise}
\begin{exercise}[Naturality of $\bang$]
  Show that $\bang$ is a natural transformation from $\Id$ to $\K1$
  where $\Id : A \mapsto A$ is the identity functor and $\K1 : A
  \mapsto 1$ the constant functor returning the terminal object.
\end{exercise}
\begin{exercise}[Naturality of pairing]
  Let $\C$ be a category that has binary products.
  \begin{enumerate}
  \item Complete the definition of the product functor $\_{\times}\_ : [\C
    \times \C, \C]$, $\_{\times}\_(A,B) = A \times B$ with its action
    $\_{\times}\_$ on
    morphisms (see \Cref{ex:hom-product}) and prove the functor laws.
  \item Formulate (if possible) a naturality statement for pairing
    $\langle\_,\_\rangle$ and prove naturality.
  \end{enumerate}
\end{exercise}

\begin{definition}[Cartesian (monoidal) category]
  A \emph{cartesian category}, more precisely, a \emph{cartesian
    monoidal category}, has finite products (including the nullary
  one).
\end{definition}

\begin{definition}[Lawvere theory]
  A \emph{Lawvere theory} is a cartesian monoidal category $T$ where
  each object is isomorphic to a power $X^n$ of a distinguished object
  $X$, called the generic object for $T$.

  A \emph{model} $A$ of $T$ is a product-preserving functor $A : [T,\Set]$
  in the sense that for each $n \in \bN$ the set $A(X^{n})$ together
  with the morphisms $A(\pi_{i}) : A(X^{n}) \rightarrow A(X)$ for $i
  \leq n$ is a product of $n$ copies of the set $A(X)$.
\end{definition}
\begin{example}
  The Lawvere theory of groups has morphism $e : X^0 \too X$ and
  $\vop : X^2 \too X$ and $\vinv : X \too X$.  A specific group can be
  represented as a model of this theory, e.g., $\Int(X) = \bZ$ and
  $\Int(e) = 0$ and $\Int(\vop)(i,j) = i + j$ and $\Int(\vinv)(i) = -i$.
\end{example}

\section{Cartesian Closed Categories}

In a cartesian category, we can represent first-order functions as
morphisms $f : A_1 \times \dots \times A_n \too B$.  To get
higher-order functions as in simply-typed lambda-calculus, we need to
be able to internalize homsets as objects.

\begin{definition}
  Given $A,B : \C$, an \emph{exponential} of $B$ to the $A$ is given
  by the following data:
  \begin{enumerate}
  \item An object $E : \C$ with
  \item a morphism $\teval : E \times A \too B$, such that
  \item for each $C$ and $f : C \times A \too B$ there is a unique $h
    : C \too E$ such that $\teval \comp (h \times \id[A]) = f$.
  \end{enumerate}
  The uniqueness of $h$ justifies the notation $h = \tcurry(f)$ (also:
  $h = \Lambda(f)$ or $h = \lambda(f)$).
  Since $E$ is unique up to isomorphism, the notation $E = B^A$ or $E
  = A \To B$ is justified.
\end{definition}
The universal property of exponentials is visualized as follows:
\[
\xymatrix@=16ex{
  C \times A \ar@{.>}[d]_{\tcurry(f) \times \tid}
      \ar[dr]^{f}\\
  B^A \times A \ar[r]^{\teval} & B \\
}
\]
\begin{exercise}
  Explain the exponentials of $\Set$ and $\Setoid$!
  Does $\Grp$ have exponentials?
\end{exercise}
\begin{exercise}
  Give an example of a preorder that has exponentials.
\end{exercise}
\begin{exercise}
  Show that the exponential is unique up to isomorphism!
\end{exercise}
\begin{exercise}[Derived laws]
   Prove these laws about exponentials:
   \begin{enumerate}
   \item $\tcurry(f) \comp h = \tcurry(f \comp (h \times \id))$.
   \item $\tcurry(\teval) = \id[B^A]$.
   \item $\tcurry(\teval \comp (f \times \id[A])) = f : C \to B^A$.
   \end{enumerate}
\end{exercise}

\begin{definition}[CCC]
  A \emph{cartesian closed category} has finite products and exponentials.
\end{definition}
\begin{exercise}
  Show that $\Cat$ is cartesian closed.
\end{exercise}

\printbibliography

\clearpage

\appendix

\section{Solutions}

\begin{solution}[\Cref{ex:pmon}]
  Given a partial monoid $(M,e,\cdot)$ let $\Ob = \Pot(M)$ and
  $m : \Hom(A,B)$ if and only if $a \cdot m$ is defined for all $a \in A$ and $m
  \cdot b$ is defined for all $b \in B$.  Then we can set $\id[A] = e$
  and $f \comp g = f \cdot g$ just as in the case for total monoids.
\end{solution}


\end{document}
